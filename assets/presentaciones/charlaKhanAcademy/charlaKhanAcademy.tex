% Created 2015-02-24 mar 08:42
\documentclass[presentation]{beamer}
\usepackage[utf8]{inputenc}
\usepackage[T1]{fontenc}
\usepackage{fixltx2e}
\usepackage{graphicx}
\usepackage{longtable}
\usepackage{float}
\usepackage{wrapfig}
\usepackage{rotating}
\usepackage[normalem]{ulem}
\usepackage{amsmath}
\usepackage{textcomp}
\usepackage{marvosym}
\usepackage{wasysym}
\usepackage{amssymb}
\usepackage{capt-of}
\usepackage{hyperref}
\tolerance=1000
\usetheme{Madrid}
\usecolortheme{}
\usefonttheme{}
\useinnertheme{}
\useoutertheme{}
\author{Alvar Maciel}
\date{\textit{<2015-02-24 mar>}}
\title{Khan Academy en elaula}
\begin{document}

\maketitle
\begin{frame}{Outline}
\tableofcontents
\end{frame}



\section{Sobre Khan Academy}
\label{sec-1}
\begin{frame}[label=sec-1-1]{¿Qué es esto?}
\begin{columns}
\begin{column}{0.6\columnwidth}
\begin{beamercolorbox}{Explicaciones}
\begin{itemize}
\item Un sitio con videos explicativos, principalmente de Ciencias y Matemáticas.
\item Un sitio con \alert{ejercicios} vinculados a esas explicaciones.
\item Una plataforma de aprendizaje.
\item Sitio: \url{http://es.khanacademy.org}
\end{itemize}
\end{beamercolorbox}
\end{column}
\begin{column}{0.4\columnwidth}
\begin{structureenv} %% Página de inicio
\begin{figure}[htb]
\centering
\includegraphics[width=.9\linewidth]{pagInicio2.png}
\caption{Página de inicio}
\end{figure}
\end{structureenv}
\end{column}
\end{columns}
\end{frame}

\begin{frame}[label=sec-1-2]{¿Para qué sirve?}
\begin{itemize}
\item Para aprender algunos temas.
\item Para repasar algunos conceptos.
\item Para ejercitar y tener un seguimiento de las ejercitaciones.
\end{itemize}
\begin{block}{¿Cómo lo uso en el aula?}
\begin{itemize}
\item Como espacio de tareas.
\item Asignación de ejercicios personalizados. No todos los chicxs hacen las mismas cosas.
\item Como recurso para las adaptaciones curriculares.
\item Con las compus de los chicos y sin las compus de los chicos.
\end{itemize}
\end{block}

\begin{block}{¿Qué hay que saber?}
\begin{itemize}
\item Como sacar las cuentas de los alumnos.
\item Cómo revisar las actividades y asignarselas a los estudiantes.
\item Nada más.
\end{itemize}
\end{block}
\end{frame}
% Emacs 24.4.1 (Org mode 8.3beta)
\end{document}
